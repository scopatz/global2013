%%%%%%%%%%%%%%%%%%%%%%%%%%%%%%%%%%%%%%%%%%%%%%%%%%%%%%%%%%%%%%%%%%%%%
%
%  This is a sample LaTeX input file for your contribution to 
%  the MC2013 conference. Modified by R.C. Martineau at INL from A. 
%  Sood at LANL, from J. Wagner ORNL who obtained the original class 
%  file by Jim Warsa, LANL, 16 July 2002}
%
%  Please use it as a template for your full paper 
%    Accompanying/related file(s) include: 
%       1. Document class/format file: mc2013.cls
%       2. Sample Postscript Figure:   figure.eps
%       3. A PDF file showing the desired appearance: template.pdf 
%    Direct questions about these files to: richard.martinea@inl.gov
%
%    Notes: 
%      (1) You can use the "dvips" utility to convert .dvi 
%          files to PostScript.  Then, use either Acrobat 
%          Distiller or "ps2pdf" to convert to PDF format. 
%      (2) Different versions of LaTeX have been observed to 
%          shift the page down, causing improper margins.
%          If this occurs, adjust the "topmargin" value in the
%          mc2013.cls file to achieve the proper margins. 
%
%%%%%%%%%%%%%%%%%%%%%%%%%%%%%%%%%%%%%%%%%%%%%%%%%%%%%%%%%%%%%%%%%%%%%


%%%%%%%%%%%%%%%%%%%%%%%%%%%%%%%%%%%%%%%%%%%%%%%%%%%%%%%%%%%%%%%%%%%%%
%\documentclass{anstrans}
\documentclass{ansconfpaper}
%
%  various packages that you may wish to activate for usage 
\usepackage{listings}
%\usepackage{color}
\usepackage{graphicx}
\usepackage{tabls}
\usepackage{hyperref}
\usepackage{verbatim}
%

%General Short-Cut Commands
\newcommand{\superscript}[1]{\ensuremath{^{\textrm{#1}}}}
\newcommand{\subscript}[1]{\ensuremath{_{\textrm{#1}}}}
\newcommand{\vin}[1]{\texttt{{#1}}}
%\newcommand{\py}[1]{\begin{lstlisting}[language=Python]{#1}\end{lstlisting}}
%\newcommand{\pyb}[0]{\begin{lstlisting}[language=Python]}
%\newcommand{\pye}[0]{\end{lstlisting}}
%\newcommand{\nuc}[2]{\superscript{#2}{#1}}
\newcommand{\nuc}[2]{{#1}-{#2}}
\newcommand{\ith}[0]{$i^{\mbox{th}}$ }
\newcommand{\jth}[0]{$j^{\mbox{th}}$ }
\newcommand{\kth}[0]{$k^{\mbox{th}}$ }
\newcommand{\us}[0]{$\mu$s }

% New definition of square root:
% it renames \sqrt as \oldsqrt
\let\oldsqrt\sqrt
% it defines the new \sqrt in terms of the old one
\def\sqrt{\mathpalette\DHLhksqrt}
\def\DHLhksqrt#1#2{%
\setbox0=\hbox{$#1\oldsqrt{#2\,}$}\dimen0=\ht0
\advance\dimen0-0.2\ht0
\setbox2=\hbox{\vrule height\ht0 depth -\dimen0}%
{\box0\lower0.4pt\box2}}



% Insert authors' names and short version of title in lines below
%
\authorHead{Anthony M. Scopatz}
\shortTitle{A dynamic, dependent type system for nuclear fuel cycle code generation}

\confTitle{GLOBAL 2013: International Nuclear Fuel Cycle Conference}
\confLocation{Salt Lake City, Utah, Sept. 29-Oct. 3, 2013}
\confPublished{on CD-ROM (2013)}

%%%%%%%%%%%%%%%%%%%%%%%%%%%%%%%%%%%%%%%%%%%%%%%%%%%%%%%%%%%%%%%%%%%%%
%
%   BEGIN DOCUMENT
%
%%%%%%%%%%%%%%%%%%%%%%%%%%%%%%%%%%%%%%%%%%%%%%%%%%%%%%%%%%%%%%%%%%%%%
\begin{document}
\lstset{language=Python}

\title{A Dynamic, Dependent Type System for Nuclear Fuel Cycle Code Generation}

\author{Anthony Scopatz}
\affil{The University of Chicago\\
  5754 S. Ellis Ave, Chicago, IL, 60637 \\
  scopatz@flash.uchicago.edu}

\maketitle

\begin{abstract}
%\raggedright

\emph{Key Words}: cyclus, Bright, fuel cycle
\end{abstract}


\section{Introduction}
\label{sec:intro}
The nuclear fuel cycle may be interpreted as a network or graph, thus 
allowing methods from formal graph theory \cite{DIE05B} to be used with 
respect to it.  Using this interpretation the edges are seen as an economic exchange 
between the nodes.  Often this exchange is modeled as a two way materials-for-money 
relationship.  However, what the nodes in the graph represent is considerably 
more complex.  

Nodes are often idealized as nuclear fuel cycle facilities (reactors, enrichment
cascades, deep geologic repositories).  Unlike the edges, though, distinct structure
is almost always needed to represent different kinds of facilities.  Intuitively 
this is because facilities within a fuel cycle implement unique, important 
tasks.  For instance, a power reactor is not substitutable for an enrichment facility
nor vice versa.  

With the advent of modern object-oriented programming languages -- and fuel cycle
simulators implemented in these languages -- it is natural to define a class 
hierarchy of facility types.  This is because classes are the syntatic sugar for type 
declaration in a language.  Furthermore, having an extendable space of types is 
critical for developing new facility models.  Such a space is neatly provided by 
object orientation.

Computationally, this state-of-affairs is sufficient provided that the facility 
models (nodes) only need to be exposed to a single fuel cycle simulator (graph).
However, if multiple simulators must be aware of the same facility type 
implementation, an  abstract representation of the facility model is needed.  
This abstraction is then
used to create concrete representations of the facility for each of the fuel cycle
simulators.  Since the facilities are often classes such an abstraction is 
known as a \emph{type system} \cite{Pierce:2002:TPL:509043}.

The requirement of one implementation for multiple simulators has recently
arisen for the Bright collection of fuel cycle facilities \cite{Scopatz2009}.
Bright's default graph representation is the Python abstract syntax tree and its
execution engine is the Python interpreter.  However, the facility models -- 
which are implemented in C++ -- must now also be hooked into the cyclus simulator.  
Cyclus uses XML as its input graph representation and a custom client-server based 
execution engine \cite{cyclus2012}.

Therefore a robust, dependent type system was developed for the nuclear fuel cycle.
This system is capable of representing any fuel cycle facility.  Types declared in 
this system can then be used to automatically generate code which binds a facility 
implementation to a simulator front end.  
Furthermore, this type system is also capable of performing both static and dynamic
verification \& validation of facility instances.
This strategy also has advantages common to code code generation techniques:
inoculation against front end interface fragility, easy 
extension to more fuel cycle simulators, \emph{etc}.
Given the motivating use case, code 
generation routines have been created for the Bright and cyclus execution engines.

The remainder of this paper proceeds as follows.  The next section covers the 
details of the type system.  Then \S \ref{sec:codegen} explains how this type system
may be used for automatic code generation.  A fuel cycle simulator that uses this
code generator is exampled in \S \ref{sec:fc}.  Lastly, some conclusions and future
work are presented in \S \ref{sec:conc}.

\section{Type System}
\label{sec:ts}

A type system is an agreed upon scaffolding on which to build, compose, and scribe 
meaning to types of information.  It is not the data itself but rather the labels
that denote different kinds of data.  For computational systems, all types are 
reducible to ordered pairs of bits -- the 0 or 1 representation of a boolean state
\cite{DBLP:journals/jsyml/Church40}.

Therefore, a \emph{dynamic type system} is one which allows additional types to 
be added to it after its initial construction.  On the other hand, a \emph{dependent 
type system} is one which allows types contructed as a function of other parameters, 
including other types.  The majority of modern programing languages provide 
syntax and compilers for both kinds of type systems.  Therefore any type system with
the goal of code generation must be both dynamic and dependent.

The xdress software package \cite{xdress} was written by the author of this paper for 
the purpose of nuclear fuel cycle code generation.  This project is a pure Python 
project which wraps C++ application programming interfaces (API).  This is because
C++ is the language in which both Bright and Cyclus facility models are implemented.

XDress provides a suite of tools in the \vin{typesystem.py} module for denoting, 
describing, and converting
between various data types and the types coming from various systems.  This is
achieved by providing canonical abstractions of various kinds of types:

\begin{itemize}
    \item Base types (int, str, float, non-templated classes)
    \item Refined types (even or odd ints, strings containing the letter \vin{`a'})
    \item Dependent types (templates such arrays, maps, sets, vectors)
\end{itemize}

All types are known by their name (a string identifier) and may be aliased with 
other names.  However, the string representation of a type is not sufficient to 
fully describe most types.  The xdress system implements a canonical form for all 
kinds of types.  This canonical form is itself hashable, being comprised only of 
Python strings, ints, and tuples.

\subsection{Canonical Forms}
\label{sec:canon}

First, consider the base types and the forms that they may take.  Base types
are fiducial.  The type system itself may not make any changes (refinements, 
template filling) to types of this kind.  They are effectively a sequence of bits.
(The job of ascribing meaning to these bits falls on external users and programming 
languages.)  Thus base types may be referred to simply by their string identifier.  
For example: \vin{`str'}, \vin{`int32'}, \vin{`float64'}, and \vin{`MyClass'}.

Aliases to these -- or any -- type names are given in the \vin{type\_aliases} 
mapping:
\begin{lstlisting}
type_aliases = {
    'i': 'int32',
    'i4': 'int32',
    'int': 'int32',
    'complex': 'complex128',
    'b': 'bool',
    }
\end{lstlisting}

Furthermore, length-2 tuples are used to denote a type and the name or flag of its
predicate.  A predicate is a function or transformation that may be applied to 
verify, validate, cast, coerce, or extend a variable of the given type.  A common 
usage is to declare a pointer or reference of the underlying type.  This is done with 
the string flags \vin{`*'} and \vin{`\&'}, e.g.,  \vin{(`char', `*')} and 
\vin{(`float64', `\&')}.

If the predicate is a positive integer, then this is interpreted as a 
homogeneous array of the underlying type with the given length.  If this length 
is zero, then the tuple is often interpreted as a scalar of this type, equivalent 
to the type itself.  The length-0 scalar interpretation depends on context.  Here 
are some examples of array types:
\begin{lstlisting}
# length-42 character array
('char', 42)  

# length-1 boolean array
('bool', 1)   

# scalar 64-bit float
('f8', 0)     
\end{lstlisting}

The next kind of type are \emph{refinement types} or \emph{refined types}.  
A refined type
is a sub-type of another type but restricts in some way what constitutes a valid 
element.  For example, given the set of all integers, the set of all positive 
integers is a refinement of the original.  Similarly, starting with all possible
strings the set of all strings starting with \vin{`A'} is a refinement.

In the xdress system, refined types are given their own unique names (e.g. 
\vin{`posint'} and \vin{`astr'}).  The type system has a mapping from all refinement
type names to the names of their super-type.  The user may refer to refinement types
simply by their string name.  However the canonical form refinement types is to use
the refinement as the predicate of the super-type in a length-2 tuple, as above:
\begin{lstlisting}
# refinement of integers to 
# positive ints
('int32', 'posint')

# refinement of strings to 
# str starting with 'A'
('str', 'astr')      
\end{lstlisting}
It is these refinement types that give the second index in the tuple its `predicate'
name.  Additionally, the predicate is used to look up the converter and validation
functions in when doing code generation or type verification.

The last kind of type are known as \emph{dependent types} or \emph{template types}, 
similar in concept to C++ template classes.  These are meta-types whose 
instantiation requires one or more parameters to be filled in by further values or
types. Dependent types may nest with themselves or other dependent types.  Fully 
qualifying a template type requires the resolution of all dependencies.

Classic examples of dependent types include the C++ template classes.  These take
other types as their dependencies.  Other cases may require only values as 
their dependencies.  For example, suppose the user wants to restrict integers to 
various ranges.  Rather than creating a refinement type for every combination of 
integer bounds, single 'intrange' type may be implemented that defines high and low 
dependencies.

The \vin{template\_types} mapping takes the dependent type names (e.g. \vin{`map'})
to a tuple of their dependency names (\vin{`key'}, \vin{`value'}).   Thus 
refined types are also accepted as tuples of the following form:
\begin{lstlisting}
('<typename>', '<dep0-name>', 
    ('dep1-name', 'dep1-type'), ..)
\end{lstlisting}
Note that template names may be reused as types of other template parameters:
\begin{lstlisting}
('name', 'dep0-name', 
    ('dep1-name', 'dep0-name'))
\end{lstlisting}

As we have seen, dependent
types may either be base types (when based off of template classes), refined types,
or both.  Their canonical form thus follows the rules above with some additional 
syntax.  The first element of the tuple is still the type name and the last 
element is still the predicate (default 0).  However the type tuples now have a
length equal to 2 plus the number of dependencies.  These dependencies are 
placed between the name and the predicate: \vin{(`<name>', <dep0>, ..., <predicate>)}.
These dependencies may be other type names or tuples themselves. 

In the simplest case, take analogies to C++ template classes::
\begin{lstlisting}
('set', 'complex128', 0)
('map', 'int32', 'float64', 0)
('map', ('int32', 'posint'), 
    'float64', 0)
('map', ('int32', 'posint'), 
    ('set', 'complex128', 0), 0)
\end{lstlisting}
Now consider the intrange type from above.  This has the following definition and
canonical form::

\begin{lstlisting}
refined_types = {('intrange', 
    ('low', 'int32'), 
    ('high', 'int32')): 'int32'}

# range from 1 -> 2
('int32', ('intrange', 
    ('low', 'int32', 1), 
    ('high', 'int32', 2)))

# range from -42 -> 42
('int32', ('intrange', 
    ('low', 'int32', -42), 
    ('high', 'int32', 42)))
\end{lstlisting}
Note that the low and high dependencies here are length three tuples of the form
\vin{(`<dep-name>', <dep-type>, <dep-value>)}.  How the dependency values end up 
being used is solely at the discretion of the implementation.  These values may
be anything, though they are most useful when they are easily convertible into 
strings in the target language.

Next, consider a \vin{`range'} type which behaves similarly to \vin{`intrange'} 
except that it also accepts the type as dependency.  This has the following 
definition and canonical form:
\begin{lstlisting}
refined_types = {('range', 'vtype', 
    ('low', 'vtype'), 
    ('high', 'vtype')): 'vtype'}
    
# integer range from 1 -> 2
('int32', ('range', 'int32', 
    ('low', 'int32', 1), 
    ('high', 'int32', 2)))    

# positive int range from 42 -> 65
(('int32', 'posint'), ('range', 
  ('int32', 'posint'),
  ('low', ('int32', 'posint'), 42),
  ('high', ('int32', 'posint'), 65)
  ))
\end{lstlisting}

\subsection{Shorthand Forms}
\label{sec:shorhand}

The canonical forms for types contain all the information needed to fully describe
different kinds of types.  However, as human-facing code, they can be exceedingly 
verbose.  Therefore there are number of shorthand techniques that may be used to 
also denote the various types.  Converting from these shorthands to the fully
expanded version may be done via the the \vin{canon(t)} function.  This function
takes a single type and returns the canonical form of that type.  The following
are operations that \vin{canon()}  performs:

\begin{itemize}
\item Base type are returned as their name: \vin{canon(`str') == `str'}.

\item Aliases are resolved: \vin{canon(`f4') == `float32'}.

\item Determines the super-type of refinements: \vin{canon(`posint') == 
    (`int32', `posint')}.

\item Applies templates:
\begin{lstlisting}
t = ('set', 'float')
canon(t) == ('set','float64',0)
\end{lstlisting}

\item Applies dependencies:
\begin{lstlisting}    
t = ('intrange', 1, 2)
canon(t) == ('int32', 
  ('intrange', 
    ('low', 'int32', 1), 
    ('high', 'int32', 2)))
\end{lstlisting}

\item Performs all of the above recursively.
\end{itemize}

These shorthands are thus far more useful and intuitive than the canonical forms 
described in \S \ref{sec:canon}.  It is therefore recommended that users and 
developers write code that uses the shorthand versions.

\section{Automatic Code Generation}
\label{sec:codegen}

Using the type system abstractions detailed in \S \ref{sec:ts} it is possible
to describe arbitrary APIs from any modern programming language.  With the aid
of external parsers and lexers, abstract syntax trees (AST) may be generated 
and visited.  In the process of visiting the API nodes of the AST, nuclear 
fuel cycle facilities can be automatically described.  Once a description is 
obtained, wrapper code from the source language to the lagunage of the fuel 
cycle simulator may also be automaticaly created.

\subsection{Descriptions}
\label{sec:desc}

The first step in automatically generating wrapper code is to have a clear and 
consice definition of API element \emph{descriptions}.  Since APIs are by definition 
independent of the implementations that they subsume the API descriptions themselves
need not comprise full ASTs.  This makes the API contract much simpler than what
is required by language syntaxes.  In C++ there are three
basic constructs which may be wrapped: variables, functions, and classes.  
The primary construct which are of interest to fuel cycle applications are classes, 
as discussed above.  However since classes may own member functions, functions are
also illustrated.

Descriptions are simply given by Python dictionaries with a specific structure.
The description structure is distinct for classes and functions and makes heavy 
use of the type system to declare the types of all needed parameters.  The top-level
items that comprise descriptions structure for classes and functions are presented 
in Tables \ref{classdesc} \& \ref{funcdesc}.

As a simple example of descriptions, consider the \vin{Reprocess} class in Bright 
which is the facility responsible for reprocessing materials. A minimally valid 
description dictionary may be seen in Figure \ref{rep_ex}. 

\begin{table*}[htbp]
\begin{center}
\caption{Class Descriptions}
\label{classdesc}
\begin{tabular}{|lp{0.8\hsize}|}
\hline
\vin{name} & string, the class name \\
\vin{parents} & list of strings, the immediate parents of the class (not grandparents) \\
\vin{namespace} & string or None, the namespace or module the class lives in. \\
\vin{attrs} & dictionary, the names of the attributes (member variables) of the
    class mapped to their types, given in the format of the type system. \\
\vin{methods} & dictionary, similar to the \vin{attrs} except that the keys are now
    function signatures and the values are the method return types.  The signatures
    themselves are tuples. The first element of these tuples is the method name.
    The remaining elements (if any) are the function arguments.  Arguments are 
    themselves length-2 or -3 tuples whose first elements are the argument names, 
    the second element is the argument type, and the third element (if present) is
    the default value.  If the return type is None (as opposed to \vin{`void'}), then 
    this method is assumed to be a constructor or destructor. \\
\vin{docstrings} & dictionary, optional, this dictionary is meant for storing 
    documentation strings.  All values are thus either strings or dictionaries of 
    strings.   Valid keys include: class, attrs, and methods.  The attrs and methods
    keys are dictionaries which may include keys that mirror the top-level keys of
    the same name. \\
\vin{extra} & dictionary, optional, this stores arbitrary metadata that may be used 
    with different backend code generators. It is not added by any auto-describer
    routine but may be inserted by the user later if needed.\\
\hline
\end{tabular}
\end{center}
\end{table*}

\begin{table*}[htbp]
\begin{center}
\caption{Function Descriptions}
\label{funcdesc}
\begin{tabular}{|lp{0.8\hsize}|}
\hline
\vin{name} & string, the function name \\
\vin{namespace} & string or None, the namespace or module the function lives in. \\
\vin{signatures} & dictionary, the keys of this dictionary are function call 
    signatures and the values are the function return types.  The signatures
    themselves are tuples. The first element of these tuples is the function name.
    The remaining elements (if any) are the function arguments.  Arguments are 
    themselves length-2 or -3 tuples whose first elements are the argument names, 
    the second element is the argument type, and the third element (if present) is
    the default value. Unlike class constuctors and destructors, the return type 
    may not be None (only \vin{`void'} values are allowed).\\
\vin{docstring} & string, optional, this is a documentation string for the function.\\
\vin{extra} & dictionary, optional, this stores arbitrary metadata that may be used
    with different backend code generators. It is not added by any auto-describer
    routine but may be inserted by the user later if needed.\\
\hline
\end{tabular}
\end{center}
\end{table*}

\onecolumn
\begin{figure}[c]
\begin{center}
\caption{Reprocessing Example}
\label{rep_ex}
\begin{lstlisting}
class_desc = {
    'name': 'Reprocess',
    'namespace': 'bright',
    'parents': ['FCComp'],
    'attrs': {'sepeff': ('map', 'int32', 'float64')},
    'methods': {
        ('Reprocess',): None,
        ('Reprocess', ('sed', ('map', 'int32', 'float64')), 
                              ('n', 'str', '""')): None,
        ('Reprocess', ('ssed', ('map', 'str', 'float64')), 
                               ('n', 'str', '""')): None,
        ('calc',): 'Material',
        ('calc', ('incomp', ('map', 'int32', 'float64'))): 'Material',
        ('calc', ('mat', 'Material')): 'Material',
        ('calc_params',): 'void',
        ('initialize', ('sed', ('map', 'int32', 'float64'))): 'void',
        ('~Reprocess',): None,
        },
    }
\end{lstlisting}
\end{center}
\end{figure}
\twocolumn

\subsection{Description \& Code Generation}
\label{sec:autodesc}

The API descriptions detailed above may serve as a contract between facilitiy 
implmentations and wrapper generators.  XDress provides a two stage mechanism 
for automating these tasks.  The first stage is automatic description generation 
and the second stage takes these descriptions and pushes them to the wrapper 
generator.  

Automatic description generation takes place by running the GCC-XML
tool \cite{gccxml} over the C++ facility implementations.  This creates an XML-based
AST for the class.  This XML representation \cite{xml} is then converted to a 
description by the xdress \vin{autodescribe.py} module.  These descriptions are
then used to create wrapper code via the xdress \vin{cythongen} tool.


\section{Relationship to the Fuel Cycle}
\label{sec:fc}

\section{Conclusions \& Future Work}
\label{sec:conc}

\setlength{\baselineskip}{12pt}

\bibliographystyle{ans}
\bibliography{refs}


\end{document}
