%%%%%%%%%%%%%%%%%%%%%%%%%%%%%%%%%%%%%%%%%%%%%%%%%%%%%%%%%%%%%%%%%%%%%
%
%  This is a sample LaTeX input file for your contribution to 
%  the MC2013 conference. Modified by R.C. Martineau at INL from A. 
%  Sood at LANL, from J. Wagner ORNL who obtained the original class 
%  file by Jim Warsa, LANL, 16 July 2002}
%
%  Please use it as a template for your full paper 
%    Accompanying/related file(s) include: 
%       1. Document class/format file: mc2013.cls
%       2. Sample Postscript Figure:   figure.eps
%       3. A PDF file showing the desired appearance: template.pdf 
%    Direct questions about these files to: richard.martinea@inl.gov
%
%    Notes: 
%      (1) You can use the "dvips" utility to convert .dvi 
%          files to PostScript.  Then, use either Acrobat 
%          Distiller or "ps2pdf" to convert to PDF format. 
%      (2) Different versions of LaTeX have been observed to 
%          shift the page down, causing improper margins.
%          If this occurs, adjust the "topmargin" value in the
%          mc2013.cls file to achieve the proper margins. 
%
%%%%%%%%%%%%%%%%%%%%%%%%%%%%%%%%%%%%%%%%%%%%%%%%%%%%%%%%%%%%%%%%%%%%%


%%%%%%%%%%%%%%%%%%%%%%%%%%%%%%%%%%%%%%%%%%%%%%%%%%%%%%%%%%%%%%%%%%%%%
\documentclass{ansconf}
%
%  various packages that you may wish to activate for usage 
\usepackage{graphicx}
\usepackage{tabls}
\usepackage{hyperref}
%

%General Short-Cut Commands
\newcommand{\superscript}[1]{\ensuremath{^{\textrm{#1}}}}
\newcommand{\subscript}[1]{\ensuremath{_{\textrm{#1}}}}
%\newcommand{\nuc}[2]{\superscript{#2}{#1}}
\newcommand{\nuc}[2]{{#1}-{#2}}
\newcommand{\ith}[0]{$i^{\mbox{th}}$ }
\newcommand{\jth}[0]{$j^{\mbox{th}}$ }
\newcommand{\kth}[0]{$k^{\mbox{th}}$ }
\newcommand{\us}[0]{$\mu$s }

% New definition of square root:
% it renames \sqrt as \oldsqrt
\let\oldsqrt\sqrt
% it defines the new \sqrt in terms of the old one
\def\sqrt{\mathpalette\DHLhksqrt}
\def\DHLhksqrt#1#2{%
\setbox0=\hbox{$#1\oldsqrt{#2\,}$}\dimen0=\ht0
\advance\dimen0-0.2\ht0
\setbox2=\hbox{\vrule height\ht0 depth -\dimen0}%
{\box0\lower0.4pt\box2}}



% Insert authors' names and short version of title in lines below
%
\authorHead{Anthony M. Scopatz}
\shortTitle{A dynamic, dependent type system for nuclear fuel cycle code generation}

\confTitle{GLOBAL 2013: International Nuclear Fuel Cycle Conference}
\confLocation{Salt Lake City, Utah, Sept. 29-Oct. 3, 2013}
\confPublished{on CD-ROM (2013)}

%%%%%%%%%%%%%%%%%%%%%%%%%%%%%%%%%%%%%%%%%%%%%%%%%%%%%%%%%%%%%%%%%%%%%
%
%   BEGIN DOCUMENT
%
%%%%%%%%%%%%%%%%%%%%%%%%%%%%%%%%%%%%%%%%%%%%%%%%%%%%%%%%%%%%%%%%%%%%%
\begin{document}

\title{A Dynamic, Dependent Type System for Nuclear Fuel Cycle Code Generation}

\author{Anthony Scopatz}
\affil{The University of Chicago\\
  5754 S. Ellis Ave, Chicago, IL, 60637 \\
  scopatz@flash.uchicago.edu}

\maketitle

\begin{abstract}
\raggedright
The nuclear fuel cycle may be intrepreted as a network or graph, thus 
allowing methods from formal graph theory \cite{DIE05B} to be used with 
respect to it.  In this case the edges are seen as economic exchange 
between the nodes.  Often this exchange is modeled as a two way materials-for-money 
relationship.  However, what the nodes represent is considerably 
more complex.  

Nodes are often idealized as nuclear fuel cycle facilities (reactors, enrichment
cascades, deep geologic repositories).  Unlike the edges, though, distinct structure
is almost always needed to represent different kinds of facilities.  Intuitively, 
this is because facilties representing a fuel cycle implement unique, important 
tasks.  For instance, a power reactor is not subistutable for an enrichment facility
nor vice versa.  (Note that nodes may also sometimes 
represent regions or institutions which also exchange various commodities, 
instead of facilities.  The following discussion also applies to them as well).

With the advent of modern object-oriented programming languages and fuel cycle
simulators implemented in these languages, it is natural to define class heirarchies
of facility types.  This is because classes are the syntatic sugar for type 
declaration in a language.  Furthermore, having an extendable space of types is 
critical for developing new facility models.

Computationally, this state-of-affairs is sufficient provided that the facility 
models (nodes) only need to be exposed to a single fuel cycle simulator (graph).
However, if multiple simulators must be aware of the same facitily type, an 
abstract represntation of the facitily model is needed.  This abstraction is then
used to create concrete representation of the facility for each of the fuel cycle
simulators.  Since the facilities are often classes such an abstraction is 
known as a \emph{type system} \cite{Pierce:2002:TPL:509043}.

The requirement of one implemention for multiple simulators has recently
arisen for the Bright collection of fuel cycle facilities \cite{Scopatz2009}.
Bright's default graph representation is the Python abstract syntax tree and its
execution engine is the Python interpreter.  However, the facility models -- 
which are implemented in C++ -- must now also be hooked into the cyclus simulator.  
Cyclus uses XML as its input graph representation and a custom client-server based 
execution engine.

Therefore a robust, dependent type system was implemented for the nuclear fuel cycle.
This system is capable of representing any fuel cycle facility.  Types declared in 
this system can then be used to automatically generate code which binds a facility 
implementation to a simulator front end.  Given the motivating use case, code 
generation routines have been created for the Bright and cyclus execution engines.

The strategy detailed here has a number of distinct advantages made possible by 
code generation abstraction. One is the 
incoluation against front end interface fragility over time. Anothere is the easy 
extension to other nuclear fuel cycle simulators.

\emph{Key Words}: cyclus, Bright, fuel cycle
\end{abstract}

\setlength{\baselineskip}{12pt}

\bibliographystyle{ans}
\bibliography{refs}


\end{document}
